\documentclass[11pt]{article} 
\usepackage[utf8]{inputenc} 


\usepackage{geometry} 
\geometry{a4paper} 

\usepackage{graphicx} 
\usepackage{booktabs} 
\usepackage{array}
\usepackage{paralist} 
\usepackage{verbatim}
\usepackage{subfig}
\usepackage{amssymb}
\usepackage{amsmath}
\usepackage{titlesec}
\usepackage{tikz}

\usetikzlibrary{trees,arrows}

\setcounter{secnumdepth}{4}


\usepackage{fancyhdr} 
\pagestyle{fancy} 
\renewcommand{\headrulewidth}{0pt} 
\lhead{}\chead{}\rhead{}
\lfoot{}\cfoot{\thepage}\rfoot{}


\usepackage{sectsty}
\allsectionsfont{\sffamily\mdseries\upshape} 


\usepackage[nottoc,notlof,notlot]{tocbibind}
\usepackage[titles,subfigure]{tocloft} 
\renewcommand{\cftsecfont}{\rmfamily\mdseries\upshape}
\renewcommand{\cftsecpagefont}{\rmfamily\mdseries\upshape} 

\renewcommand{\P}{\mathbf{P}}
\newcommand{\NP}{\mathbf{NP}}
\newcommand{\RP}{\mathbf{RP}}
\newcommand{\CoRP}{\textbf{Co-}\mathbf{RP}}
\newcommand{\PPT}{\mathbf{PPT}}
\newcommand{\BPP}{\mathbf{BPP}}
\newcommand{\IP}{\mathbf{IP}}
\newcommand{\PSPACE}{\mathbf{PSPACE}}

\newcommand{\N}{\mathbb{N}}

\newcommand{\mcH}{\mathcal{H}}
\newcommand{\mcU}{\mathcal{U}}



\title{CS 183 Notes}
\author{Stephen Kelman\\ Rafi Ostrovsky, Eli Jaffe}


\begin{document}
\section{Lecture 9: More on Zero-Knowledge Proofs}

\subsection{ZKP for GNI}

Now, we return to graph non-isomorphism. Recall the interactive proof from before:
\begin{enumerate}
\item \(V\) generates a random bit \(b\) and a permutation \(\pi\) on \(G_b\)
\item \(V\) sends \(H=\pi(G_b)\)
\item \(P\) sends back a bit \(b'\)
\item \(V\) accepts if \(b=b'\)
\end{enumerate}
And this is repeated \(k\) times. \bigskip

Now, this might seem like zero-knowledge, but we must remember that zero-knowledge is trying to account for a dishonest verifier, so we have to consider all the ways the verifier might be dishonest, especially if the prover is being honest.\medskip

In particular, what if \(H\) is not generated properly. What if \(H\) is found somewhere else, and the verifier just sends it to see what it gets back from the prover? Well, if the prover is being honest and tests it against each of \(G_0\) and \(G_1\), and sends back a bit \(b'\), the verifier has obviously just learned something, since he didn't know whether \(H\) was even isomorphic to either of the two graphs before!\medskip

So, we're gonna have to add an intermediate step before \(P\) sends back \(b'\), where \(V\) proves that it knows what \(b\) is, from which it will follow that \(H\) was generated properly, so \(P\) sending \(b'\) will not be giving away any new information to \(V\).

\subsubsection{Proof of Knowledge}
This is called a ``proof of knowledge", and here's what we're going to do:
\begin{enumerate}
\item \(V\) generates two graphs, \(H_0\) and \(H_1\), isomorphic to \(G_0\) and \(G_1\), respectively.
\item \(V\) picks a random bit \(b\), and sends the ordered pair \((H_b,H_{b'})\)
\item Proof of Knowledge (repeat k times, possibly in parallel, for soundness):
\begin{enumerate}
\item \(V\) sends two graphs \(C_0,C_1\), in any order.
\item \(P\) sends back a bit \(d\)
\item If \(d=1\), then \(V\) must show isomorphisms mapping one of the \(C\)'s to \(G_0\), and the other to \(G_1\). If \(d=0\), then \(V\) must map the \(C\)'s to the \(H\)'s.
\end{enumerate}
\item \(P\) sends a bit \(b'\)
\item \(V\) accepts if \(b=b'\)
\end{enumerate}

If \(G_0\sim G_1\), we can see that the only way for \(V\) to pass the proof-of-knowledge with non-negligible probability is to actually have \(H_0\sim G_0\) and \(H_1\sim G_1\). Otherwise, on any given pair of \(C\)'s, \(V\) wouldn't have an answer with probability \(1/2\).\smallskip

And, the proof-of-knowledge doesn't reveal anything. Suppose that it somehow revealed which \(H\) corresponds to which \(G\). Then, \(P\) must have found an isomorphism between \(C\)'s of different trials. But since all the isomorphic graphs are either identical to both \(G_0\sim H_0\) or both \(G_1\sim H_1\), this is just as difficult as actually figuring out what \(b\) is. So using this proof to cheat is just as difficult as actually solving for \(b\) as intended. So, the original soundness of the protocol remains.\smallskip

It is easy to see that since an honest prover can still employ the same strategy as before, this protocol also remains perfectly complete.\bigskip

Finally, we have that the protocol is zero-knowledge by the following strategy of a simulator. \smallskip

Note that the verifier takes no input before sending the first pair of \(C\)'s. So all randomness up to this point can be completely controlled to be the same, every time, by the simulator. In particular, the simulator can run the verifier once, and then ask for \(d=1\) to get a mapping of the \(C\)'s to the \(G\)'s. Then, it can start over and run the verifier again, with everything the same, except this time it asks for \(d=0\) to get a mapping of these same \(C\)'s to the \(H\)'s. Then, since the \(C\)'s are the same, this reveals which \(H\) corresponds to which \(G\), so the simulator can then proceed and correctly guess \(b\).

\subsection{ZKP of ``AND", ``OR"}
If an honest prover wants to prove that \(X\) is true and \(Y\) is true, then it must prove that both statements are true. So, we don't risk anything by simply providing two separate ZKP's, one for \(X\) and one for \(Y\).\bigskip

But what about proving \(X\) is true \emph{OR} \(Y\) is true? If only one is true, then we don't want to reveal which one is false, but we want to show that at least one is true.\smallskip

If we are working with \(GNI\) and \(GI\), we have strategies to make this work.

\subsubsection{GNI ``OR" ZKP}
This protocol turns out to have a very simple idea. If we want to prove \(G_0\not\sim G_1\) OR \(H_0\not\sim H_1\), then we will simply run the ZKP's for both in parallel. We will have \(V\) choose the same bit \(b\) for both of them. Then, since the prover only has to reveal the bit, and the bit is the same for both, nothing is revealed about which graphs are isomorphic or not.\medskip

One interesting thing to consider is that if we somehow know our prover is honest, the verifier may cheat and figure out which pairs are actually isomorphic by picking different bits. However, this isn't actually feasible for a dishonest verifier, since this would allow a dishonest prover to be right with probability \(1\), so the soundness guarantee is ruined for the verifier if it tries this. 

\subsubsection{GI ``OR" ZKP}
Assuming \(G_0\sim G_1\) or \(H_0\sim H_1\), consider the following protocol:
\begin{enumerate}
\item \(P\) generates bits \(d_1,d_2\), and permutations \(\pi_1,\pi_2\).
\item \(P\) sends graphs \(C=\pi_1(G_{d_1})\) and \(D=\pi_2(H_{d_2})\).
\item \(V\) sends a random bit \(b\)
\item \(P\) sends permutations \(\pi_1': C\to G_{b_1}\) and \(\pi_2': C\to H_{b_2}\), where \(b_1\oplus b_2=b\).
\end{enumerate}

It is important to note that \(b_i\) and \(d_i\) are not necessarily the same, and that the only way for \(P\) to produce \(b_i\ne d_i\) is to know an isomorphism between \(G_0\sim G_1\) or \(H_0\sim H_1\).\medskip

So, if \(P\) is dishonest, it will fail with probability \(1/2\). And if it is honest, it knows one of the isomorphisms between the \(G\)'s or \(H\)'s, allowing it to produce any \(b\) via XOR. So this protocol has strong soundness (\(1/2^k\) breaking probability) and perfect completeness.\medskip

And, since \(V\) only ever sees an isomorphism between \(C\) and one of the \(G\)'s, and between \(D\) and one of the \(H\)'s, there is no way for it to figure out which of the statements is true, except by figuring it out on its own. So, the protocol seems intuitively ZK.\smallskip

 In order to create a simulation, we may find that we have to add the same sort of commitment scheme as for the original GI ZKP, and once we do that, there should be no problem creating a simulation just as we did for the original GI ZKP.



\subsection{Every \(\NP\) Problem has a Zero-Knowledge Proof}
Recall that Graph 3-coloring is NP-complete. That is, every problem in NP can be converted to an instance of a graph 3-coloring problem in polytime.\smallskip

It follows that if we are able to show a general protocol for zero-knowledge proofs of graph 3-coloring, we will then have a way to prove any NP statement with a zero-knowledge proof. The only assumption we will need to make is a commitment scheme.\bigskip

The basic idea is:
\begin{enumerate}
\item [0.] \(P\) has a 3-coloring of a graph
\item \(P\) permutes the colors on the graph and commits all the colors
\item \(V\) sends back one edge
\item \(P\) opens the colors of the two vertices incident to that edge
\item \(V\) accepts if the two opened colors are different
\end{enumerate}
Steps 1-4 are repeated as many times as necessary to have a satisfactory level of soundness.

\newpage
We can see easily that we have perfect completeness.\smallskip

And, for a single run of the protocol a dishonest verifier will fail with probability \(\ge\frac{1}{|E|}\), where \(E\) is the set of edges in the graph. Thinking in terms of Bernoulli trials, we expect to catch a dishonest verifier after \(|E|\) trials, and over \(k\cdot|E|\) trials, we expect a dishonest verifier to fail \(k\) times. \smallskip

Similarly, a dishonest verifier gets away with being dishonest with probability \(\le\frac{|E|-1}{|E|}\) on any given trial. So thinking probabilistically we can see that the probability of a dishonest verifier getting away with being dishonest decreases exponentially as the number of trials increases linearly.\smallskip

So, we have computational soundness.\medskip

The simulator is not difficult to construct either; the strategy is very similar to our original GI simulator. If the verifier is honest, we can simply get the edge before coloring the graph, which allows us to get it right every time. If the verifier is dishonest, we can do the rewind-and-cut strategy, and we expect to only have to do this \(|E|\) times for each trial we want to produce, meaning the full simulation is still poly-time.

\end{document}